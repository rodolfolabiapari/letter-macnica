%%%%%%%%%%%%%%%%%%%%%%%%%%%%%%%%%%%%%%%%%
% Thin Formal Letter
% LaTeX Template
% Version 1.11 (8/12/12)
%
% This template has been downloaded from:
% http://www.LaTeXTemplates.com
%
% Original author:
% WikiBooks (http://en.wikibooks.org/wiki/LaTeX/Letters)
%
% License:
% CC BY-NC-SA 3.0 (http://creativecommons.org/licenses/by-nc-sa/3.0/)
%
%%%%%%%%%%%%%%%%%%%%%%%%%%%%%%%%%%%%%%%%%

%----------------------------------------------------------------------------------------
%	DOCUMENT CONFIGURATIONS
%----------------------------------------------------------------------------------------

\documentclass{letter}

% Adjust margins for aesthetics
\addtolength{\voffset}{-0.5in}
\addtolength{\hoffset}{-0.3in}
\addtolength{\textheight}{2cm}
\usepackage[utf8]{inputenc}
\usepackage[portuguese]{babel}
\usepackage{url}

%\longindentation=0pt % Un-commenting this line will push the closing "Sincerely," to the left of the page

%----------------------------------------------------------------------------------------
%	YOUR NAME & ADDRESS SECTION
%----------------------------------------------------------------------------------------

\signature{Rodolfo Labiapari Mansur Guimarães.} % Your name for the signature at the bottom

\address{Instituto de Ciências Exatas e Biológicas\\Universidade Federal de Ouro Preto\\Rua Quatro, 786 \\ Ouro Preto, Bauxita, Minas Gerais 35400-000 \\ \url{rodolfolabiapari@decom.ufop.br}} % Your address and phone number

%----------------------------------------------------------------------------------------

\begin{document}

%----------------------------------------------------------------------------------------
%	ADDRESSEE SECTION
%----------------------------------------------------------------------------------------

\begin{letter}{Suporte Macnica DHW\\ Rua Patricio Farias, 131, Lj 01, Itacorubi \\ Florianópolis, Santa Catarina 88034-132} % Name/title of the addressee

%----------------------------------------------------------------------------------------
%	LETTER CONTENT SECTION
%----------------------------------------------------------------------------------------

\opening{\textbf{Prezados,}}
 
Como conversado com o senhor Moreira (\url{gustavo.leao@macnicadhw.com.br}) e Andrade (\url{fernando.andrade@macnicadhw.com.br}), tive a infelicidade de ligar uma fonte de 12V na entrada de energia externa da placa Mercurio IV e como consequência parte de seu circuito não responde mais. Em primeira instância, pensei que havia queimado-a totalmente, mas depois de alguns dias percebi que ela ainda ligava mas sem resposta de seu componente Blaster-USB.

Trocando vários e-mails com o grupo de suporte técnico, percebi que existe uma super temperatura no chip \textit{U7} localizado próximo ao componente de \textit{Ethernet}. Também houve alta temperatura em \textit{U10} e também em \textit{X1} mas receio que esses sejam resultantes da alta temperatura do \textit{U7}.

Solicitaram a troca do chip \textit{U7}, mas não possuo habilidade e nem equipamentos para realizar tal processo com chip tão pequeno quanto aquele e com isso envio-lhes a placa para que vocês possam fazer um orçamento de custo de reparo.

Desde já agradeço a assistência pelo apoio e paciência em testar cada componente comigo via e-mail.



\vspace{2\parskip} % Extra whitespace for aesthetics
\closing{Atenciosamente,}
\vspace{2\parskip} % Extra whitespace for aesthetics


%----------------------------------------------------------------------------------------

\end{letter}
 
\end{document}